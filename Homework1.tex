\documentclass[10pt,a4paper]{article}
\usepackage{homework} % See homework.sty %
\usepackage{graphicx}
\usepackage{float}
\usepackage{array}
\usepackage[export]{adjustbox}  
\usepackage{listings}
\newcolumntype{M}[1]{>{\centering\arraybackslash}m{#1}}

\author{1155193237 - Yu Ching Hei\\email: chyu2@cse.cuhk.edu.hk}
\title{CENG3420 - Computer Organization \& Design\\Homework 1}
\pagestyle{fancy}
\renewcommand{\headrulewidth}{0pt}
\fancyhf{}
\cfoot{\thepage}
\date{Date: \today}
\begin{document}
\maketitle

\question{(10\%) 
This is a question about integrated circuit cost. Assume that a wafer contains 4096
dies and a die has 0.15 defects on average, please answer the following sub-questions.
\begin{enumerate}
    \item Calculate the yield of this wafer. (5\%)
    \item Assume that you wanted to spend 10 millions HKD on manufacturing, how much
    money can you save for manufacturing the same number of dies if the average defects
    of a die can be reduced to 0.075? (5\%)
\end{enumerate}
}
\ans{
\begin{enumerate}
    \item $$\text{Yield} = \frac{1}{[1 + (0.15 \div 2)]^2} = \frac{1600}{1849} = 86.5\%$$
    \item Lets assuem the cost per wafer is \$C\\
            Then, $$\text{Cost per die} = \frac{C}{\text{Die per wafer} \times \text{Yield}}$$
            Therefore, 
            \begin{equation} \label{equ1}
                \begin{split}
                \text{Saved money} & = 10,000,000 \times (\frac{\text{Yield}_{0.15}}{\text{Yield}_{0.075}} - 1) \\
                 & = 10,000,000 \times \{\frac{[1 + (0.15 \div 2)]^2}{[1 + (0.075 \div 2)]^2} - 1\} \\
                 & = 10,000,000 \times \frac{507}{6889} \\
                 & = \$735,956 (\text{round off to the nearest dollar})
                \end{split}
            \end{equation}
\end{enumerate}
}
\question{(10\%) 

Suppose we developed a new processor that has 70\% of the capacitive load of the older processor. 
Further, it can reduce voltage 10\% compared to previous generation, which results in a 20\% 
shrink in frequency. What is the impact on dynamic power? Give the ratio of $\frac{\text{Power}_{\text{new}}}{\text{Power}_{\text{old}}}$
}
\ans{
Given the power consumption of a processor can be computed by the following formula
$$\text{Power} =  \text{Capacitive load} \times \text{Voltage}^2 \times \text{Frequency}^1 $$
So we have the ratio
$$\text{Ratio} = \frac{70\% \times (1 - 10\%)^2 \times (1 - 20\%)}{1} 
= 0.7 \times 0.9^2 \times 0.8 = 0.4536 $$
}
\question{(20\%)
We have an int (32 bits) array named arr0. 
The pointer of arr0's first element stored in register a0. 
Please answer the following questions.
\begin{enumerate}
    \item How to put the 5th element of arr0 to register t1? (5\%)
    \item How to calculate t1 + 32 and store the result in register t2? (5\%)
    \item Find an efficient way to calculate t2 / 32 and t2 \% 32. Please store the results in t3 and t4, 
    respectively. Note that / is an integer division and \% is the modulo operation. 
    (hint: using shift and logical operations) (10\%)
\end{enumerate}
}
\ans{
    \begin{enumerate}
        \item 
            \begin{verbatim}
                lw t1, 16(a0)
            \end{verbatim}
        \item 
            \begin{verbatim}
                addi t2, t1, 32
            \end{verbatim}
        \item 
            \begin{itemize}
                \item Division:
                    \begin{verbatim}
                        srli t3,t2,5
                    \end{verbatim}
                \item Modulo:
                    \begin{verbatim}
                        andi t4,t2,0x1F
                    \end{verbatim}
            \end{itemize}
    \end{enumerate}
}
\pagebreak

\question{(20\%)
We have an int (32 bits) array named arr1. The pointer to arr1’s first element stored 
in register a0. We also have the registers t1 = 0xAAABCDEF, t2 = 0xF8000000

Please answer the following questions:
\begin{enumerate}
    \item What is the value of t3 for the following sequence of instructions? (5\%)
    \begin{center}
        \texttt{slli t3, t1, 12\\
        srli t3, t3, 12}
    \end{center}
    \item What is the value of t3 for the following sequence of instructions? (5\%)
    \begin{center}
        \texttt{slli t3, t2, 4\\
        srai t3, t3, 4}
    \end{center}
    \item Write a piece of assembly program to: (10\%)
    \begin{itemize}
        \item Store the result of t1 XOR t2 to register t3 (3\%)
        \item Store t3 to the first element of arr1; (3\%)
        \item Store the lowest 8 bits of t3 to the third element of arr1. (4\%)
    \end{itemize}
        
\end{enumerate}
}
\ans{
    \begin{enumerate}
        \item 0xBCDEF
        \item 0xF8000000
        \item 
        \begin{itemize}
            \item xor t3, t1, t2
            \item sw t3, 0(a0)
            \item andi t3, t3, 0xFF\\
                sw t3, 12(a0)
        \end{itemize}
    \end{enumerate}
}

\question{(20\%)
Consider the following RISC-V instructions: \\
\texttt{li t1, 5\\
li t2, 1\\
li t3, 0\\
LOOP:\\
beq t1, t3, DONE\\
mul t2,t2,t1\\
addi t1, t1, -1\\
jal ra, LOOP\\
DONE:\\
\# end of the program} 
\begin{enumerate}
    \item How many times is the loop executed (between LOOP and DONE)? (5\%)
    \item List the value of t2 at each loop iteration. (10\%)
    \item What does this program do? (5\%)
\end{enumerate}
}
\ans{
    \begin{enumerate}
        \item 5 full loop with one extra bew t1, t3, DONE instruction executed
        \item 1, 5, 20, 60, 120, 120
        \item factorial of t1
    \end{enumerate}
}
\pagebreak
\question{(20\%)
This is a question about using stack. Write RISC-V instructions to implement the following functionalities.
\begin{enumerate}
    \item Reserve a stack area that can save three words of data (1 word = 4 bytes). (5\%)
    \item Save the return address in ra, data in a0 and data in a1 to the stack. (5\%)
    \item Restore the return address and data in the stack to corresponding registers. (5\%)
    \item Free the reserved stack space. (5\%)
\end{enumerate}
}
\ans{
    \begin{verbatim}
        addi sp, sp, -12
        sw ra, 0(sp)
        sw a0, 4(sp)
        sw a1, 8(sp)
        lw a1, 8(sp)
        lw a0, 4(sp)
        lw ra, 0(sp)
        addi sp, sp, 12
    \end{verbatim}
}
\end{document}